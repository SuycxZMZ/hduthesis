% !TEX program = xelatex
% !TeX encoding = UTF-8
% !TeX spellcheck = en_US
% !BIB program = bibtex

\documentclass[twoside]{hduthesis}
% 如果不需要双面打印时插入的空白页,可以将twoside选项改成oneside

% ======= 根据自己需要加载相应宏包 ==========
\usepackage{mathrsfs}       % 提供mathscr: 花体
\usepackage{amsmath}        % AMSLaTeX宏包 用来排出更加漂亮的公式
\usepackage{amssymb}        % ams数学符号
\usepackage{physics}        % 物理常用符号
\usepackage{threeparttable} % 表格加脚注
\usepackage{booktabs}       % 表格,横的粗线;\specialrule{1pt}{0pt}{0pt}
\usepackage{longtable}      % 支持跨页的表格。
\usepackage{blindtext}      % 产生测试用的文字,实际使用中可删除
% =================================

% =================================
% 定义所有的图片文件在 figures 子目录下
% 如果图片放在主目录,则可以注释下面的命令
\graphicspath{{figures/}}
% =================================

% =================================
% 论文封面需要的基本信息配置
% 注意:
%   1. '\hdusetup'配置里面不要出现空行 (很重要,不然会编译出错!)
%   2. 后面带*的是相应的英文参数。比如author是中文名,author*是英文名
% =================================
\hdusetup{
    % 中文标题在封面上分为两行显示,分别为titleA和titleB。如果一行放得下,titleB留空白即可。
    titleA  = {杭州电子科技大学硕士学位论文},
    titleB  = {\LaTeX{} 模板使用示例文档 v\version},
    title*  = {An Introduction to \LaTeX{} Thesis Template of Hangzhou Dianzi
              University v\version},
    % 专业
    discipline  = {物理学},
    discipline* = {Physics},
    % 姓名
    author  = {张三},
    author* = {San Zhang},
    % 指导教师,中文姓名和职称之间空格分开,下同
    supervisor  = {李四\ 教授},
    supervisor* = {Professor Si Li},
    % 日期,如果不指定则默认显示编译当月
    % date = {2022年4月},
    % data* = {April, 2022}
}
% =================================

\begin{document}

\maketitle

\frontmatter
% 摘要
% !TeX root = ../thuthesis-example.tex

% 中英文摘要和关键字

\begin{abstract}
硕士论文摘要的字数一般为500字左右,博士论文摘要的字数为800--1000 字。
内容包括研究工作目的意义、 研究方法、所取得的结果和结论,应突出本论文的创造性成果或新见解,语言精炼。
英文摘要的内容应与中文摘要相对应,要符合英语语法,语句通顺,文字流畅。
中、英文摘要均应列出关键词,关键词是为了便于文献标引, 从该学位论文中选取出来用以表示全文主题内容信息款目的单词或术语,一般选取3至8个,关键词之间用顿号分隔。
  
    \hdusetup{
        keywords = {关键词1、 关键词2、 关键词3、 关键词4、 关键词5},
    }
\end{abstract}

\begin{abstract*}
  An abstract of a dissertation is a summary and extraction of research work and contributions.
  Included in an abstract should be description of research topic and research objective, brief introduction to methodology and research process, and summary of conclusion and contributions of the research.
  An abstract should be characterized by independence and clarity and carry identical information with the dissertation.
  It should be such that the general idea and major contributions of the dissertation are conveyed without reading the dissertation.

  An abstract should be concise and to the point.
  It is a misunderstanding to make an abstract an outline of the dissertation and words “the first chapter”, “the second chapter” and the like should be avoided in the abstract.

  Keywords are terms used in a dissertation for indexing, reflecting core information of the dissertation.
  An abstract may contain a maximum of 5 keywords, with semi-colons used in between to separate one another.

  % Use comma as separator when inputting
    \hdusetup{
        keywords* = {keyword 1, keyword 2, keyword 3, keyword 4, keyword 5},
    }
\end{abstract*}


% 中文目录
\tableofcontents 

% 正文部分
\mainmatter

% 包含章节
% !TeX root = ../main.tex

\chapter{模版说明}

\section{格式标准}

该模版按照~\cite{hdy2012311}要求的格式制作。
参考文献的排版按照标准~\cite{gbt7714-2005}执行。

\section{使用方法}

下面介绍该模版的使用方法.

\subsection{文件说明}

该模版包含文件结构如表~\ref{tab:file}所示。

\begin{table}[htp]
    \centering
    \caption{文件说明}
    \begin{tabular}{ll}
      \toprule
      文件名          & 描述                         \\
      \midrule
      hduthesis.cls   & 模板文件                     \\
      HDULOGO.pdf     & 学校标志的图片 \\
      main.tex & 主文件 \\
      ref.bib & BibTeX文件    \\
      data/ & 包含具体TeX文件的文件夹 \\
      figures/ & 默认用来存放图片的文件夹 \\
      data/chap01.tex & 第一章的TeX文件(建议每章一个TeX文件)\\
      data/abstract.tex & 中英文摘要 \\
      data/acknowledgements.tex & 致谢 \\
      data/appendix.tex & 附录 \\
      \bottomrule
    \end{tabular}
    \label{tab:file}
  \end{table}
  
\subsection{编译命令}

推荐的编译命令为``\cmd{latexmk -xelatex -synctex=1 main.tex}''。
``\cmd{latexmk}''命令的运行需要系统安装有Perl解释器。
可以使用命令``\cmd{latexmk -C}''来删除编译产生的文件,可以使用命令``\cmd{latexmk -c}''来删除编译产生的临时文件。

\section{插图}

图片通常在 \env{figure} 环境中使用 \cs{includegraphics} 插入,如图~\ref{fig:example} 的源代码。
建议矢量图片使用 PDF 格式,比如数据可视化的绘图;
照片应使用 JPG 格式;
其他的栅格图应使用无损的 PNG 格式。
注意,LaTeX 不支持 TIFF 格式;EPS 格式已经过时。

\begin{figure}
    \centering
    \includegraphics[width=0.5\linewidth]{example-image-a.pdf}
    \caption{示例图片标题}
    \label{fig:example}
\end{figure}
  
\chapter{test chapter}
\blinddocument % 产生测试文本

\clearpage

\phantomsection
\addcontentsline{toc}{chapter}{参考文献}
\bibliographystyle{gbt7714-2005-numerical}
\bibliography{ref}  % 参考文献BibTeX文件名称,默认为ref.bib

% 附录,所发文章与参与项目
% !TeX root = ../main.tex

\chapter*{附录}
\addcontentsline{toc}{chapter}{附录}

\begin{center}
    作者在读期间发表的学术论文及参加的科研项目
\end{center}

发表的学术论文:
\begin{enumerate}
    \item 论文信息。
\end{enumerate}

  
% 致谢
% !TeX root = ../main.tex

\begin{acknowledgements}
    后记:对某方面进行补充,或对给予各类资助、指导和协 助完成研究工作,以及提供各种条件的单位及个人表示感谢。致 谢应实事求是,切忌浮夸与庸俗之词。
\end{acknowledgements}
 

\end{document}
