% !TeX root = ../main.tex

\chapter{模版说明}

\section{格式标准}

该模版按照~\cite{hdy2012311}要求的格式制作。
参考文献的排版按照标准~\cite{gbt7714-2005}执行。

\section{使用方法}

下面介绍该模版的使用方法.

\subsection{文件说明}

该模版包含文件结构如表~\ref{tab:file}所示。

\begin{table}[htp]
    \centering
    \caption{文件说明}
    \begin{tabular}{ll}
      \toprule
      文件名          & 描述                         \\
      \midrule
      hduthesis.cls   & 模板文件                     \\
      HDULOGO.pdf     & 学校标志的图片 \\
      main.tex & 主文件 \\
      ref.bib & BibTeX文件    \\
      data/ & 包含具体TeX文件的文件夹 \\
      figures/ & 默认用来存放图片的文件夹 \\
      data/chap01.tex & 第一章的TeX文件(建议每章一个TeX文件)\\
      data/abstract.tex & 中英文摘要 \\
      data/acknowledgements.tex & 致谢 \\
      data/appendix.tex & 附录 \\
      \bottomrule
    \end{tabular}
    \label{tab:file}
  \end{table}
  
\subsection{编译命令}

推荐的编译命令为``\cmd{latexmk -xelatex -synctex=1 main.tex}''。
``\cmd{latexmk}''命令的运行需要系统安装有Perl解释器。
可以使用命令``\cmd{latexmk -C}''来删除编译产生的文件,可以使用命令``\cmd{latexmk -c}''来删除编译产生的临时文件。

些或类似词语。

\section{插图}

图片通常在 \env{figure} 环境中使用 \cs{includegraphics} 插入,如图~\ref{fig:example} 的源代码。
建议矢量图片使用 PDF 格式,比如数据可视化的绘图;
照片应使用 JPG 格式;
其他的栅格图应使用无损的 PNG 格式。
注意,LaTeX 不支持 TIFF 格式;EPS 格式已经过时。

\begin{figure}
    \centering
    \includegraphics[width=0.5\linewidth]{example-image-a.pdf}
    \caption{示例图片标题}
    \label{fig:example}
\end{figure}
  